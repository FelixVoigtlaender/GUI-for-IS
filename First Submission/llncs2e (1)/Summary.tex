% This is samplepaper.tex, a sample chapter demonstrating the
% LLNCS macro package for Springer Computer Science proceedings;
% Version 2.20 of 2017/10/04
%
\documentclass[runningheads]{llncs}
%
\usepackage{graphicx}
% Used for displaying a sample figure. If possible, figure files should
% be included in EPS format.
%
% If you use the hyperref package, please uncomment the following line
% to display URLs in blue roman font according to Springer's eBook style:
% \renewcommand\UrlFont{\color{blue}\rmfamily}

\begin{document}
%
\title{Contribution Title\thanks{Supported by organization x.}}
%
%\titlerunning{Abbreviated paper title}
% If the paper title is too long for the running head, you can set
% an abbreviated paper title here
%
\author{Felix Voigtländer\inst{1}}
%
\authorrunning{F. Author et al.}
% First names are abbreviated in the running head.
% If there are more than two authors, 'et al.' is used.
%
\institute{RWTH Aachen, Aachen 52062, Germany}
%
\maketitle              % typeset the header of the contribution
%
\begin{abstract}
The abstract should briefly summarize the contents of the paper in
150--250 words.

\keywords{Prototyping \and RAD \and Another keyword.}
\end{abstract}
%
%
%
\section{Summary}

Prototyping has many advantages, with the technique users are more involved in development, requirements are investigated more clearly, 
end users are more commited to the project and cleaner code is used with many more advantages \cite{ref_proto_prac}. But a project using prototyping is hard to manage,
especially using older management models like the waterfall modell, which works best with clearly defined and static requirements.
Yet prototyping results in rapidly shifting requirements \cite{ref_proto_prac} \cite{ref_proto_ui} which makes older modells unsuitable and
inefficient when using this approach.

Traditional modells in particular the waterfall model is vulnerable to change and inefficient when using customer feedback resulting in massive reworks 
at the end of projects. With the use of prototyping focusing on customer feedback and their involvement has become increasingly important. 
For which the waterfall model is not suitable, which is one of many reasons to turn to agile.

Agile is flexible and thus can easily react to feedback from users and stakeholders. Users are able to bring in their feedback during 
all stages in development, instead of unclear and static documentation at the beginning of a project, leading to better user satisfaction and 
a better product. Due to the emphasis on early and fast prototyping users are able to give direct feedback, which is more easily implemented at such early stages 
in development. Agile also aims to be self improving making processes and development more efficient and adaptable to change.

Kanban is an agile tool for visualizing workflow and managing it. It limits work in progress, improves the flow of parallel tasks in development
and is a great tool for management. This model does not force organizational, structural or process changes, instead change happens iteratively.

Rapid Prototyping in combination with agile methodoliges and the use of the Kanban model improve the negative impact rapid prototyping has on managing projects.


\section{List of chapter titles}

\subsection{Introduction}
\subsection{Prototyping}
\subsection{Waterfall modell}
\subsection{Agile Development}
\subsection{Kanban}
\subsection{Discussion}
\subsection{Conclusion}



%
% ---- Bibliography ----
%
% BibTeX users should specify bibliography style 'splncs04'.
% References will then be sorted and formatted in the correct style.
%
% \bibliographystyle{splncs04}
% \bibliography{mybibliography}
%
\begin{thebibliography}{8}

\bibitem{ref_proto_prac}
Paul Beynon-Davies, Douglas Tudhope, Hugh Mackay: Information systems prototyping in practice. 
Journal of Information Technology  \textbf{14}(1), 107--120 (1999) 

\bibitem{ref_proto_health}
R. Lenz, K.A. Kuhn: Towards a continuous evolution and adaptation of
information systems in healthcare. 
International Journal of Medical Informatics   \textbf{2004}(73), 75--89

\bibitem{ref_proto_ui}
Dirk Bäumer, Walter R. Bischofberger, Horst Lichter, Heinz Züllighoven: User Interface Prototyping - Concepts, Tools, and Experience.
Proceedings - International Conference on Software Engineering 1996, DBLP \doi{10.1109/ICSE.1996.493447}

\bibitem{ref_waterfall}
Dr. Winston W. Rovce: MANAGING THE DEVELOPMENT OF LARGE SOFTWARE SYSTEMS (1987)

\bibitem{ref_agilemanifesto}
The Agile Manifesto, \url{https://agilemanifesto.org/}. Last accessed 15
Nov 2020

\bibitem{ref_featurecreep}
Feature Creep, \url{https://www.agile-academy.com/de/agiles-lexikon/feature-creep/
}. Last accessed 15
Nov 2020

\bibitem{ref_management_kanban}
Eric Brechner: Agile Project Management with Kanban. Microsoft Press,
Redmond, Washington 98052-6399 (2015)

\end{thebibliography} 
\end{document}
